\documentclass[20pt, a1paper, portrait]{tikzposter}
\usepackage[utf8]{inputenc}
\usepackage[T1]{fontenc}
\usepackage{graphicx}
\usepackage{xcolor}
\usepackage{amsmath}

% Boje (iz pozadine logotipa)
\definecolor{ftncolor}{HTML}{42959D}

% Podešavanja teme
\usetheme{Default}
\usecolorstyle[colorPalette={white, ftncolor, ftncolor}]{Default}

% Prilagođavanje boja za header i blokove
\colorlet{titlefgcolor}{white}
\colorlet{titlebgcolor}{ftncolor}
\colorlet{blocktitlefgcolor}{white}
\colorlet{blocktitlebgcolor}{ftncolor}
\usetitlestyle{Filled}
\useblockstyle{Default}

% Prilagođavanje headera da logo bude desno i sve centrirano
\makeatletter
\settitle{ \vbox{
    \color{titlefgcolor}
    \vspace*{2em}
    \begin{minipage}{0.7\textwidth}
        \raggedright
        {\bfseries \Huge \@title \par}
        \vspace*{0.8em}
        {\huge \textit{Analiza akustičkih karakteristika govora za detekciju nivoa depresije} \par}
        \vspace*{1.2em}
        {\LARGE \@author \quad | \quad \@institute \par}
    \end{minipage}
    \hfill
    \begin{minipage}{0.25\textwidth}
        \raggedleft
        \includegraphics[width=\linewidth]{ftnlogo.jpeg}
    \end{minipage}
    \vspace*{2em}
}}
\makeatother

\title{
\parbox{0.8\linewidth}{
Predviđanje nivoa depresije kod ljudi na osnovu audio zapisa
}
}
\author{Veselin Roganović, SV 36/2022}
\institute{Fakultet tehničkih nauka, Novi Sad}

\begin{document}

\maketitle[width=1.0\textwidth]

\begin{columns}
    \column{0.33}
    \block{Uvod}{
        Depresija predstavlja jedan od najvećih izazova modernog društva. Pravovremena detekcija je ključna, ali često teška čak i za stručnjake.
        Ovaj projekat istražuje mogućnost automatske procene nivoa depresije na osnovu \textbf{akustičkih karakteristika govora}, bez analize sadržaja izgovorenih reči.
    }

    \block{Skup podataka}{
        Korišćen je \textbf{DAIC-WOZ} skup podataka koji sadrži:
        \begin{itemize}
            \item Audio zapise kliničkih intervjua.
            \item \textbf{PHQ-8} skorove (0-24) kao ciljnu vrednost.
            \item Ukupno 140 uzoraka (80\% trening, 10\% val, 10\% test).
        \end{itemize}
    }

    \block{Pretprocesiranje}{
        \begin{itemize}
            \item Resampling na 16kHz i normalizacija amplitude.
            \item Ekstrakcija MFCC koeficijenata (u vremenskim prozorima).
        \end{itemize}
    }

    \column{0.33}
    \block{Metodologija}{
        Primenjena su dva različita pristupa:
        \begin{enumerate}
            \item \textbf{SVR (Support Vector Regression):}
                \begin{itemize}
                    \item Ulaz: MFCC matrice (vremenski prozori).
                    \item Arhitektura: SVR sa RBF kernelom (C=1, epsilon=0.01, gamma='scale')
                \end{itemize}
            \item \textbf{CNN + LSTM:}
                \begin{itemize}
                    \item Ulaz: MFCC matrice (vremenski prozori).
                    \item Arhitektura: Konvolucioni slojevi za prostorna obeležja praćeni LSTM slojevima za sekvence i potpuno povezanim slojem.
                \end{itemize}
        \end{enumerate}
    }

    \block{Arhitektura CNN+LSTM}{
        \begin{itemize}
            \item \textbf{CNN:} 2D slojevi sa BatchNormalization i Dropout-om.
            \item \textbf{LSTM:} 2 sloja sa 64 skrivene jedinice.
            \item \textbf{Gubitak:} Kombinovani MAE i Pearson loss.
        \end{itemize}
    }

    \column{0.33}
    \block{Rezultati}{
        \begin{center}
            {\Large \textbf{Validation}} \par
            \vspace{0.5em}
            \begin{tabular}{l|c|c|c}
                \textbf{Model} & \textbf{MAE} & \textbf{RMSE} & \textbf{Prsn} \\
                \hline
                SVR & 4.29 & \textbf{4.98} & \textbf{0.52} \\
                CNN+LSTM & \textbf{3.99} & 5.02 & 0.45 \\
            \end{tabular}
        \end{center}
        \vspace{1em}
        \begin{center}
            {\Large \textbf{Test}} \par
            \vspace{0.5em}
            \begin{tabular}{l|c|c|c}
                \textbf{Model} & \textbf{MAE} & \textbf{RMSE} & \textbf{Prsn} \\
                \hline
                SVR & \textbf{4.37} & \textbf{6.17} & \textbf{0.27} \\
                CNN+LSTM & 5.51 & 7.20 & 0.05 \\
            \end{tabular}
        \end{center}
        \vspace{0.5em}
        SVR pobedjuje u svim kategorijama, osim za MAE na validacionom skupu.
    }

    \block{Zaključak}{
        Rezultati potvrđuju da akustička obeležja nose značajne informacije o mentalnom stanju govornika.
    }

    \block{Korišćeni alati}{
        \footnotesize
        \texttt{Python, PyTorch, Librosa, Scikit-Learn} \\
    }
\end{columns}

\end{document}
